\chapter{Einleitung}

Die Teilchenphysik beschäftigt sich mit dem Verständnis der Physik auf elementarster Ebene. Dazu gehört die Beschreibung und Vorhersage der Elementarteilchen sowie deren Wechselwirkungen untereinander. Über die letzten Jahrzehnte ist dabei das sogenannte Standardmodell der Teilchenphysik entstanden, welches bis heute die beste Beschreibung in dieser Hinsicht liefert. Dennoch existieren Phänomene, die sich durch das Standardmodell nicht beschreiben lassen: die Existenz von dunkler Materie oder Neutrinooszillationen sind Beispiele dafür. Die ständige Überprüfung der Vorhersagen also, sowie die Suche nach Physik, die über das Standardmodell hinaus geht sind Aufgaben von Physikern an Teilchenbeschleunigern wie dem LHC (Large Hadron Collider) am CERN. \\
Diese Arbeit beschäfigt sich mit der Suche nach dem \textit{lepon-flavor}-verletzenden (LFV) Zerfall $\symup{J}/\symup{\Psi}\rightarrow e^{\pm}\mu^{\mp}$, also Physik jenseits des Standardmodells.
Dazu werden Daten von dem Experiment \textit{LHCb} aus dem soundso run bei derunder Luminosität analysiert. Über einen kinetisch vergleichbaren Zerfall, nämlich $\symup{J}/\symup{\Psi}\rightarrow \mu^{\pm}\mu^{\mp}$ wird hierbei eine zum Signalkanal relative Normierungskonstante $\alpha$ bestimmt. Da der Kontrollkanal ein im SM erlaubter Prozess mit großer Zerfallsbreite ist, lassen sich hierfür deutlich größere Datenmengen nehmen, was zu einer Verringerung statistischer und messungsbedingter Ungenauigkeiten führt. Die Normierungskonstante dient in der Analyse des Signalkanals der Abschätzung eines oberen Limits für die Zerfallsbreite des LFV Zerfalls. Zur Bestimmung dieser Variable werden die Daten für $\symup{J}/\symup{\Psi}\rightarrow \mu^{\pm}\mu^{\mp}$ selektiert und analysiert [+ blabla].
