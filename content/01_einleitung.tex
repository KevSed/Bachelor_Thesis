\chapter{Einleitung}

Die Teilchenphysik beschäftigt sich mit dem Verständnis der Physik auf elementarster Ebene. Dazu gehört die Beschreibung der Elementarteilchen sowie deren Wechselwirkungen untereinander. Über die letzten Jahrzehnte ist dabei das sogenannte Standardmodell der Teilchenphysik entstanden, welches bis heute die beste Beschreibung in dieser Hinsicht liefert. Dennoch existieren Phänomene, die sich durch das Standardmodell nicht beschreiben lassen: die Existenz von dunkler Materie, Gravitation oder Neutrinooszillationen sind Beispiele dafür. Die ständige Überprüfung der Vorhersagen, sowie die Suche nach Physik, die über das Standardmodell hinaus geht sind Aufgaben von Physikern an Teilchenbeschleunigern wie dem LHC (Large Hadron Collider) der Europäischen Organisation für Kernforschung am CERN. \\
Diese Arbeit beschäfigt sich mit der Suche nach dem \textit{lepon-flavor}-verletzenden (LFV) Zerfall \signal, also Physik jenseits des Standardmodells.
Dazu werden Daten von dem Experiment LHCb aus dem soundso run bei derunder Luminosität analysiert. Über einen Kontrollzerfall, nämlich \kontroll, welcher eine ähnliche Topologoie wie der Signalzerfall zeigt, wird hierbei eine zum Signalkanal relative Normierungskonstante $\alpha$ bestimmt. Da der Kontrollkanal ein im SM erlaubter, gut vermessener Prozess mit großem Verzweigungsverhältnis (hohe statistische Genauigkeit) ist, lassen sich hierfür in statistischer und messungsbedingter Ungenauigkeit deutlich verringerte Aussagen treffen. Die Normierungskonstante dient in der Analyse des Signalkanals der Abschätzung eines oberen Limits für die Zerfallsbreite des LFV Zerfalls. Zur Bestimmung dieser Variable werden die Daten für \kontroll selektiert und analysiert [+ blabla]. \cite{ba-maik}
