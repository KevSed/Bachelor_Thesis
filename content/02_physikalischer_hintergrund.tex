\chapter{Physikalischer Hintergrund}
\label{chap:2}
%
\section{Das Standardmodell der Teilchenphysik}
%
Das Standardmodell (SM) der Teilchenphysik \footnote{Diese Ausführung basiert auf jenen in  \cite{griffiths, HalzenMartin}.} beschreibt den Aufbau der Materie, sowie ihre Wechselwirkung auf elementarer Ebene. Sie stellt eine seit 1978 \cite{griffiths} über viele Jahrzehnte auf der speziellen Relativitätstheorie sowie der Quantentheorie erwachsene und vielfältig getestete Theorie dar. Allgemein werden zunächst zwei Arten von Teilchen unterschieden: Fermionen (halbzahliger Spin $s=\sfrac{1}{2}$) und Bosonen (ganzzaliger Spin $s=1$). Die Fermionen im Standardmodell sind in drei Generationen von Quarks, sowie drei Generationen von Leptonen unterteilt, wie sie in Abbildng~\ref{fig:particles} aufgeführt sind. Die Leptonengenerationen bestehen hierbei aus einem ganzzahlig (in Einheiten der Elementarladung) geladenen punktförmigen Lepton ($e$, $\mu$, $\tau$), sowie den dazugehörigen ungeladenen und masselosen Neutrinos $\nu_e$, $\nu_\mu$, $\nu_\tau$.
%
\begin{figure}
  \centering
      \includegraphics[width=0.6\textwidth]{Plots/SM.pdf}
  \caption{Die Elementarteilchen im Stadardmodell der Teilchenphysik.}
  \label{fig:particles}
\end{figure}
%
Auch die Quarks gliedern sich in drei Generationen. Diese erfolgt über die Eigenschaften der Teilchen: die Quarks lassen sich in \textit{up-artige} Quarks mit Ladung $\sfrac{2}{3}$, sowie \textit{down-artige} mit Ladung $\sfrac{-1}{3}$ einteilen. Es gilt für die Darstellung in Abbildung~\ref{fig:particles}, dass die Teilchenmassen zwischen den Generationen von links nach rechts zunehmen.\\
Im Standardmodell unterscheidet man zwischen drei Wechselwirkungen der Elementarteilchen untereinander: die starke Wechselwirkung zwischen farbgeladenen Teilchen, die schwache Wechselwirkung an welcher alle Elementarteilchen teilnehmen, sowie die elektromagetische Wechselwirkung, welcher nur elektrisch geladene Teilchen unterliegen. Die letzten beiden lassen sich im Rahmen des SM zur elektroschwachen Wechselwirkung vereinigen.
Die Farbladung in der starken Wechselwirkung beschreibt das Konzept einer Quantenzahl deren Existenz zur theoretischen Umsetzung des sogenannten \textit{confinement} dient. \textit{Confinement} meint hierbei die Tatsache, dass alle elementaren Teilnehmer der starken Wechselwirkung nur in "farbneutralen" (z.B. Farbe + Antifarbe) Zuständen frei existieren; freie Quarks lassen sich, da sie eine von null verschiedene Farbladung tragen also nicht beobachten.\\
%
Die Übertragung der Wechselwirkungen findet über die in Abbildung~\ref{fig:particles} genannten Bosonen statt. Bei der starken Wechselwirkung sind dies die Gluonen ($g$). Sie tragen eine Farbladung und einen ganzzahligen Spin $s=1$. Die Austauschteilchen der elektroschwachen Wechselwirkung sind die Photonen ($\gamma$) für den elektromagnetischen Teil, sowie für die schwache Wechselwirkung das neutrale $\symup{Z}$-Boson und die geladenen $\symup{W^{\pm}}$-Bosonen. \\
%
Aus den in Abbildung~\ref{fig:particles} aufgeführten Quarks (bis auf das \textit{top}-Quark, dessen Lebensdauer zu gering ist \cite{pdg}) existieren über Kombination mehrere so genannte Hadronen - also aus Quarks zusammengesetzte Teilchen. Hierbei unterscheidet man die aus Quark und Antiquark bestehenden Mesonen und die aus drei Quarks (Antiquarks) bestehenden Baryonen. Zu den Mesonen zählt beispielsweise auch das $J/\!\symup{\psi}$ mit einem Quarkinhalt von $(c\bar{c})$, während das Proton ein prominenter Vertreter der Baryonen ist. Die meisten der aus den sechs Quarks sowie deren Antiteilchen gebildeten Hadronen sind nicht stabil, sodass sie über eine der oben genannten Wechselwirkungen in andere Hadronen sowie Leptonen zerfallen. Ähnliches lässt sich auch durch Streuprozesse oder Kollisionen erzielen, wie sie beispielsweise am LHC stattfinden.\\
%
Im Standardmodell sind bei all solchen Zerfällen diverse Erhaltungsgrößen zu beachten. Neben den klassischen Größen, wie etwa Energie- oder Impulserhaltung sind für die verschiedenen Wechselwirkungen auch einige Quantenzahlen im Teilchenzerfall invariant. Eines der fundamentalen Konzepte ist die \textit{lepton-flavor}-Erhaltung.
%
\section{\texorpdfstring{LFV und der Zerfall $\signal$}{Jpsi to eµ}}
%
Wie in der Einleitung bereits erwähnt, ist der Signalzerfall $\signal$ ein im Standardmodell verbotener Zerfall, weil er die Erhaltung des \textit{lepton-flavor} verletzt. Jedem Lepton wird hierbei gemäß der in Abbildung~\ref{fig:particles} aufgeführten Generationen eine Quantenzahl zugeordnet (der \textit{lepton-flavor}). Elektronen oder Elektronneutrinos besitzen beispielsweise die Quantenzahl $l_e=1$, aber $l_\mu=0$, während Myonen $l_\mu=1$ und $l_e=0$ tragen. Antiteilchen unterscheiden sich jeweils im Vorzeichen (negativ). So lassen sich die Anfangs- und Endzustände von Teilchenzerfällen auf die Erhaltung des \textit{lepton-flavor} überprüfen; der Zerfall $\signal$ verstößt hierbei offensichtlich gegen diese Erhaltung.\\
%
Es gibt einige theoretische Vorhersagen über Mechanismen und Möglichkeiten der LFV; die meisten davon beschreiben Physik jenseits des Standardmodells. Ein im Standardmodell über Neutrinooszillation möglicher Zerfall ist in dem folgenden Feynmann-Diagramm dargestellt.
%
\begin{figure}[H]
  \centering
  \begin{tikzpicture}
    \tikzfeynmanset{ every vertex={ black, dot}}
    \begin{feynman}
      \vertex (a1) {\(\mu\)};
      \vertex[right=1cm of a1] (a2);
      \vertex[right=2cm of a2] (a3);
      \vertex[right=1cm of a3] (a4) {\(e\)};
      \vertex at ($(a2)!1cm!(a3)!0.75!90:(a3)$) (d);
      \vertex[above=1.5cm of a4] (c){\(\gamma\)};
      \diagram*{
        (a1) -- [fermion](a2),
        %(a2) -- [fermion](a3),
        %(a2) -- [plain, edge label'=\(\nu_\mu\), insertion={0.5}] (a3),
        (a2) -- [plain, edge label'=\(\nu_\mu/\nu_e\), insertion={0.5}](a3),
        (a2) -- [boson, quarter left, edge label=\(W\)] (d) [square dot] -- [boson,quarter left, edge label=\(W\)] (a3),
        (a3) -- [fermion](a4),
        (d) --  [photon, quarter left](c)
          };
    \end{feynman}
  \end{tikzpicture}
  \label{fig:lfv_nu}
\end{figure}
%
Da die Masse der Neutrinos nicht verschwindend ist, können sogennante Oszillationen in andere \textit{lepton-flavors} stattfinden. Über diesen Mechanismus ist ein Zerfall möglich, der an jedem Vertex \textit{lepton-flavor}-erhaltend ist. Da die Massen der Neutrinos allerdings als sehr klein abgeschätzt werden können, sind die Beiträge dieses Zerfalls zu gering, als dass sie experimentell nachgewiesen werden können \cite{neutrino}. Die Wahrscheinlichkeit für diesen Zerfall ist proportional zum Verhältnis von W-Masse und Neutrinomasse hoch vier (Gleichung~\ref{eq:BRmuega} \cite{neutrino}), was zu einer theoretischen Abschätzung für die Zerfallsbreite führt, die so gering ist (~$10^{-54}$), dass diese nicht messbar ist. Experimentelle Evidenz deutete daher auf Physik jenseits des Stadardmodells hin.
%
\begin{equation}
  \label{eq:BRmuega}
  \text{BR}(\mu\rightarrow e\gamma)=\frac{3\alpha}{32\pi}\left|\sum_i U_{\mu i}^*U_{ei}\frac{m^2_{\nu_i}}{M^2_W}\right|^2\; .
\end{equation}
%
Andere theoretische Beschreibungen gehen etwa von Zerfällen über ein $Z'$-Boson aus \cite{zprime}. Das dazugehörige Feynman-Diagramm ist in Abbildung~\ref{fig:Zprime} dargestellt.
%
\begin{figure}
  \begin{subfigure}{0.48\textwidth}
    \centering
    \feynmandiagram[horizontal=a to b] {
    i1 [particle=\(c\)] -- [fermion] a -- [fermion] i2 [particle=\(\bar{c}\)],
    a -- [scalar, edge label=\(Z'\)] b,
    f1 [particle=\(e^{+}\)] -- [fermion] b -- [fermion] f2 [particle=\(\mu^{-}\)],
    };
    \caption{}
    \label{fig:Zprime}
  \end{subfigure}
  \begin{subfigure}{0.48\textwidth}
    \centering
    \begin{tikzpicture}
      \tikzfeynmanset{ every vertex={ black, dot}}
      \begin{feynman}
        \vertex (a1) {\(c\)};
        \vertex[right=2cm of a1] (a2);
        \vertex[right=2cm of a2] (a3){\(\mu^{-}\)};
        \vertex[above=2cm of a1] (a4){\(\bar{c}\)};
        \vertex[right=2cm of a4] (a5);
        \vertex[right=2cm of a5] (a6){\(e^{+}\)};
        \diagram*{
                (a1) -- [fermion](a2),
                (a2) -- [fermion](a3),
                (a2) -- [scalar, edge label=\(LQ\)](a5),
                (a5) -- [fermion](a4),
                (a6) -- [fermion](a5);
                };
        \end{feynman}
      \end{tikzpicture}
      \caption{}
      \label{fig:lepto}
  \end{subfigure}
  \caption{Der Signalzerfall über ein Z' (a) bzw. über ein Leptoquark (b).}
\end{figure}
%
Auch Prozesse über aus der Theorie der Supersymmetrie postulierte Teilchen sind nicht ausgeschlossen \cite{susy_gut2}. Andere theoretische Erklärungsansätze benötigen etwa sogenannte Leptoquarks \cite{leptoq}. Diese Bosonen ($s=1$) würden sowohl Farbladung, als auch einen \textit{lepton-flavor} tragen. Eine Kopplung an beide Teilchensorten, sowie Änderungen ihrer Quantenzahlen wären damit möglich (Abbildung~\ref{fig:lepto}).\\
Der hier untersuchte Zerfall $\kontroll$ stellt mit einer Wahrscheinlichkeit von $\SI{5,986(33)}{\percent}$ einen kleinen Teil aller gemessenen $\symup{J}/\symup{\psi}$-Zerfälle dar \cite{pdg}. Allerdings liegt diese im Vergleich zu der zum Signalzerfall $\signal$ nach oben abgeschätzten mit $1,6\cdot 10^{-7}$ (bei einem Konfidenzlevel von $\SI{90}{\percent}$) einige Größenordnungen höher \cite{pdg}. Dieser Zerfall eignet sich aufgrund seiner ähnlichen Topologie (Zerfall des $\symup{J}/\symup{\psi}$ in zwei Leptonen) und hohen Statistik als Kontrollkanal. Ein Vergleich mit dem Signalzerfall ermöglicht demnach die Eliminierung einiger systematischer Fehlerquellen, welche bei den erwarteten Größenordnungen der Signalkandidaten eine signifikante Rolle spielen.
