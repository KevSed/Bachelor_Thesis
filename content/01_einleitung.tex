\chapter{Einleitung}
%
Die Teilchenphysik beschäftigt sich mit dem Verständnis der Physik auf elementarster Ebene. Dazu gehört die Beschreibung der Elementarteilchen sowie deren Wechselwirkungen untereinander. Über die letzten Jahrzehnte ist dabei das sogenannte Standardmodell der Teilchenphysik entstanden, welches bis heute die beste Beschreibung in dieser Hinsicht liefert. Dennoch existieren Phänomene, die sich durch das Standardmodell nicht beschreiben lassen: die Existenz von dunkler Materie, Gravitation oder Neutrinooszillationen sind Beispiele dafür. Die ständige Überprüfung der Vorhersagen, sowie die Suche nach Physik, die über das Standardmodell hinaus geht, sind Aufgaben von Physikern an Teilchenbeschleunigern wie dem LHC (Large Hadron Collider) der Europäischen Organisation für Kernforschung CERN. \\
Ziel dieser Arbeit ist es zunächst, zu überprüfen, ob eine Analyse des Zerfalls $\signal$ mit Daten des LHCb-Detektors möglich ist. Dazu wird die Bestimmung einer oberen Ausschlussgrenze vorgenommen. Eine parallel durchgeführte Untersuchung beschäfigt sich mit der Suche nach dem \textit{lepon-flavor}-verletzenden (LFV) Zerfall $\signal$, also Physik jenseits des Standardmodells \cite{ba-maik}. Um die Ergebnisse dieser Signalanalyse statistisch sicher physikalisch [?] interpretieren zu können, beschäftigt sich diese Arbeit mit der Untersuchung und Analyse eines Kontrollkanals: $\kontroll$. [Umformulierung]
Dazu werden im Jahre 2012 am LHCb Experiment bei einer Schwerpunktsenergie von $\SI{8}{\tera\electronvolt}$ gemessene Daten, die einer Luminosität von $\SI{2}{\femto\barn^{-1}}$ entsprechen, analysiert. Da der Kontrollkanal ein im Stadardmodell erlaubter, gut vermessener Prozess mit großem Verzweigungsverhältnis (hohe statistische Genauigkeit) ist, lassen sich hierfür in statistischer und messbedingter Ungenauigkeit deutlich verringerte Aussagen treffen. Die Normierungskonstante dient in der Analyse des Signalkanals der Abschätzung eines oberen Limits für das Verzweigungsverhältnis des LFV Zerfalls. Zur Bestimmung dieser Variable werden die Daten für $\kontroll$ selektiert und analysiert.\\
Die Struktur dieser Arbeit ist viergeteilt: Zunächst wird in Kapitel~\ref{chap:2} das zugrundeliegende Standardmodell der Teilchenphysik, sowie der physikalische Hintergrund des betrachteten Zerfalls beschrieben. Kapitel~\ref{chap:3} beschäftigt sich mit der Konstruktion und Funktion des LHC und des LHCb Detektors. Es wird erläuert welche Detektoren zur Eerstellung der verwendeten Datensätze verwendet werden. Der Hauptteil der Arbeit - die Analyse - wird in Kapitel~\ref{chap:4} behandelt. Dazu wird die parallel vorgenommene Analyse des Signalkanals $\signal$ kurz erläutert und anschließend die Ermittlung der Normierungskonstante vorgenommen. In einem letzten Schritt werden die beiden Ergebnisse kombiniert und eine obere Abschätzung für die Zerfallsbreite des Signalkanals bestimmt (Kapitel~\ref{chap:5}).
