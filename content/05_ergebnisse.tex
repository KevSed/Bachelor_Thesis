\chapter{Ergebnisse der Analyse}
\label{chap:5}
%
Die in dieser Arbeit über die Analyse des Zerfalls $B^{+}\rightarrow\jpsi(\rightarrow\mu\mu) K^{+}$ bestimmte Normierungskonstante beträgt $\alpha=(1,3\,\pm\,0,4)\cdot10^{-5}$. Daraus folgt über die in der Analyse \cite{ba-maik} ermittelten Größen eine obere Abschätzung für das Verzweigungsverhältnis des Zerfalls $\signal$ von
%
\begin{equation}
  \mathcal{BR}(\signal)<(9,2\,\pm\,3,4)\cdot10^{-5} \, .
  \label{eq:ergebnisss}
\end{equation}
%
Diese Arbeit stellt in Kombination mit der Analyse \cite{ba-maik} eine erste Untersuchung des verbotenen Zerfalls $\signal$ mit
Daten des LHCb-Detektors dar. Aus diesem Grunde ist zunächst die Frage zu klären, ob die durchgeführten Messungen für eine signifikante
Aussage genügen. Aus den hier beschriebenen Ergebnissen lässt sich ableiten, dass eine solche Analyse mit Daten des LHCb möglich ist. Auch der hier verwendete Kontrollkanal eignet sich zur Bestimmung der Normierungskonstante. Die Annahmen in der Bestimmung des Korrekturfaktors in Abschnitt~\ref{sec:norm} und die daraus resultierenden Ungenauigkeiten lassen sich mit Hilfe eines in der Selektion dem in \cite{ba-maik} verwendeten entsprechenden Datensatz beispielsweise vermeiden. Das hier ermittelte Ergebnis lässt sich so mit der passenden Selektion noch optimieren.\\
Das bestimmte Verzweigungsverhältnis liegt nicht unterhalb der momentan besten Abschätzung \cite{BESIII} von $1,5\cdot10^{-7}$ ($\SI{90}{\percent}$ Konfidenzlevel). Allerdings beschränkt der enge zeitliche Rahmen einer Bachelorarbeit die Optimierungsmöglichkeiten dieser Analyse stark. Dennoch kann aus dieser Analyse und der parallel durchgeführten \cite{ba-maik} eine Abschätzung der Größenordnung $10^{-5}$ getroffen werden. Eine weitergehende Optimierung der Analysen verspricht also eine weitere Verbesserung dieser Abschätzung.
%
\section{Ausblick}
%
Die in dieser Arbeit durchgeführte Bestimmung der Normierungskonstante lässt sich in einigen Belangen unabhängig von der Detektorleistung
weiterführend genauer ausführen. Die in Kapitel~\ref{chap:4} beschriebene Selektion könnte auf eine effizientere Methodik der Suche nach angemessenen Schnitten untersucht werden. Auch die in Abschnitt~\ref{sec:massenfit} beschriebene Funktion zur Modellation des Signales lässt sich mit einer differenzierteren Parametrisierung erweitern. Hierzu ist eine Studie zur Bestimmung geeigneter Startwerte und Parameterintervalle für die Ausgleichsrechnung sinnvoll. Die Signalmodellation lässt sich wie in Abschnitt~\ref{sec:massenfit} kurz beschrieben durch etwaige besser an das Problem angepasste Funktionen verbessern. Die Ipatia-Funktion ist hier ein Beispiel. Außerdem lässt sich der Untergrund
durch komplexere Funktionen als dem hier verwendeten exponentiellen Zusammenhang exakter darstellen. Dies würde selbstverständlich auch
Einfluss auf die Ergebnisse der Signalmodellation nehmen. Abgesehen davon lässt sich die Aussagekraft der hier bestimmten oberen Grenze auch durch einen größeren Datensatz verstärken, da vor Allem in der Signalanalyse statistische und auch systematische Fehler bei der geringen Zahl an erwarteten Signalkandidaten eine wichtige Rolle spielen.
Trotz der eben beschriebenen Optimierungsmöglichkeiten stellt diese Bachelorarbeit eine gute Abschätzung für das obere Limit des Verzweigungsverhältnisses dar und zeigt, dass eine Analyse des Zerfalls $\signal$ mit Daten des LHCb möglich ist.
