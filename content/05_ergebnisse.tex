\chapter{Ergebnisse der Analyse}
\label{chap:5}
%
Diese Arbeit stellt in Kombination mit der Analyse \cite{ba-maik} eine erste Untersuchung des verbotenen Zerfalls $\signal$ mit
Daten des LHCb-Detektors dar. Aus diesem Grunde ist zunächst die Frage zu klären, ob die durchgeführten Messungen für eine signifikante
Aussage genügen. Aus den hier beschriebenen Ergebnissen lässt sich ableiten, dass eine solche Analyse [sinnvoll/oder nicht, hängt von Ergebnissen ab] möglich ist.
Die Aufnahme von für diesen Zerfall verwertbare Daten mit dem LHCb Detektor profitiert von der enorm hohen Datenrate des LHC. Allerdings
erweist sich auch als schwierig, aufgrund der Tatsache, dass der Detektor zwar sehr zuverlässige Daten zur Identifikation der Myonen liefert,
allerdings diesen Grad der Genauigeit im Bereich der Elektronenidentifikation nicht aufweist. Elektronen verursachen im Gegensatz zu Myonen aufgrund ihrer vergleichsweise
geringen Masse einen signifikanten Beitrag an Bremsstrahlung. Die hierdurch verursachten Abweichungen müssten signifikant korrigiert werden.
Die erhaltenen Ergebnisse stellen im Vergleich zu der bisher bekannten Abschätzung eine [...] dar.

\section{Ausblick}
%
Die in dieser Arbeit durchgeführte Bestimmung der Normierungskonstante lässt sich in einigen Belangen unabhängig von der Detektorleistung
weiterführend genauer durchführen. Die in Kapitel~\ref{chap:4} beschriebene Selektion könnte auf eine effizientere Methodik der Suche nach angemessenen
Schnitten untersucht werden. Die in dieser Arbeit verwendeten Funktionen zur Modellation des Signales könnten mit einer differenzierteren
Parametrisierung an die Daten angepasst oder über etwaige besser an das Problem angepasste Funktionen ersezt werden. Außerdem lässt sich der Untergrund
durch komplexere Funktionen als den hier verwendeten exponentiellen Zusammenhang eventuell exakter darstellen. Dies würde selbstverständlich auch
Einfluss auf die Ergebnisse der Signalmodellation nehmen. Die gesamte verwendete Methodik kann auf systematische Fehler hin eingehend überprüft und
damit das Ergebnis genauer bestimmt werden. Abgesehen davon lässt sich die Aussagekraft der hier bestimmten oberen Grenze auch durch einen größeren
Datensatz verstärken, da vorallem in der Signalanalyse statistische und auch systematische Fehler bei der geringen Zahl an erwarteten Signalkandidaten
eine wichtige Rolle spielen.
