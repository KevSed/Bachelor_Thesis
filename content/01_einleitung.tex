\chapter{Einleitung}
%
Die Teilchenphysik beschäftigt sich mit dem Verständnis der Physik auf elementarster Ebene. Dazu gehört die Beschreibung der Elementarteilchen sowie deren Wechselwirkungen untereinander. Über die letzten Jahrzehnte ist dabei das sogenannte Standardmodell der Teilchenphysik entstanden, welches bis heute die beste Beschreibung in dieser Hinsicht liefert. Dennoch existieren Phänomene, die sich durch das Standardmodell nicht beschreiben lassen: die Existenz von dunkler Materie, Gravitation oder Neutrinooszillationen sind Beispiele dafür. Die ständige Überprüfung der Vorhersagen, sowie die Suche nach Physik, die über das Standardmodell hinaus geht, sind Aufgaben von Physikern an Teilchenbeschleunigern wie dem LHC (Large Hadron Collider) der Europäischen Organisation für Kernforschung CERN. \\
Eine parallel durchgeführte Untersuchung beschäftigt sich mit der Suche nach dem \textit{lepon-flavor}-verletzenden Zerfall $\signal$, also Physik jenseits des Standardmodells \cite{ba-maik}. Das Ziel dieser Arbeit ist es, eine obere Ausschlussgrenze für das Verzweigungsverhältnis dieses Zerfalls zu bestimmen. Um diese Ausschlussgrenze in Kombination mit den Ergebnissen der Signalanalyse zu bestimmen, beschäftigt sich diese Arbeit mit der Bestimmung einer Normierungskonstante $\alpha$. Diese erfolgt über die Analyse des Kontrollkanals $B^{+}\rightarrow\jpsi(\rightarrow\mu\mu)K^{+}$. Der Kontrollkanal ist ein im Stadardmodell erlaubter, gut vermessener Prozess mit großem Verzweigungsverhältnis (hohe statistische Genauigkeit), weswegen sich hierfür in statistischer und messbedingter Ungenauigkeit deutlich verringerte Aussagen treffen lassen. Da für diesen Kanal bisher keine Analyse auf Basis von Daten des LHCb-Detektors existiert, wird über die Bestimmung der oberen Ausschlussgrenze auch untersucht, ob eine Analyse dieses Zerfalls mit Daten des LHCb-Detektors möglich ist.
Dazu werden im Jahre 2012 am LHCb Experiment bei einer Schwerpunktsenergie von $\SI{8}{\tera\electronvolt}$ gemessene Daten, die einer Luminosität von $\SI{2}{\femto\barn^{-1}}$ entsprechen, analysiert. \\
Die Struktur dieser Arbeit ist viergeteilt: Zunächst wird in Kapitel~\ref{chap:2} das zugrundeliegende Standardmodell der Teilchenphysik, sowie der physikalische Hintergrund des betrachteten Zerfalls beschrieben. Kapitel~\ref{chap:3} beschäftigt sich mit der Konstruktion und Funktion des LHC und des LHCb Detektors. Es wird erläuert welche Detektoren zur Erstellung der verwendeten Datensätze verwendet werden. Der Hauptteil der Arbeit - die Analyse - wird in Kapitel~\ref{chap:4} behandelt. Dazu wird die Zahö der erwarteten Signalkandidaten für den Zerfall $\kontroll$ bestimmt und mit den Ergebnissen der parallel vorgenommenne Analyse für $\signal$ die Ermittlung der Normierungskonstante vorgenommen. In einem letzten Schritt werden die beiden Ergebnisse kombiniert und eine obere Abschätzung für die Zerfallsbreite des Signalkanals bestimmt (Kapitel~\ref{chap:5}).
