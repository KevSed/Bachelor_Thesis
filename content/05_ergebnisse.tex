\chapter{Ergebnisse der Analyse}
\label{chap:5}
%
Diese Arbeit stellt in Kombination mit der Analyse \cite{ba-maik} eine erste Untersuchung des verbotenen Zerfalls $\signal$ mit
Daten des LHCb-Detektors dar. Aus diesem Grunde ist zunächst die Frage zu klären, ob die durchgeführten Messungen für eine signifikante
Aussage genügen. Aus den hier beschriebenen Ergebnissen lässt sich ableiten, dass eine solche Analyse [sinnvoll/oder nicht] möglich ist.
Die Aufnahme von für diesen Zerfall verwertbare Daten mit dem LHCb Detektor profitiert von der enorm hohen Datenrate des LHC. Allerdings
erweist sich auch als schwierig, aufgrund der tatsache, dass der Detektor zwar sehr zuverlässige Daten zur Identifikation der Myonen liefert,
allerdings Schwächen im Bereich der Elektronidentifikation aufweist. Elektronen verursachen im Gegensatz zu Myonen aufgrund ihrer vergleichsweise
geringen Masse einen signifikanten Beitrag an Bremsstrahlung. Die hierdurch verursachten Abweichungen können nur unzureichend korrigiert werden,
Große Datenmengen.
Die erhaltenen Ergebnisse stellen im Vergleich zu der bisher bekannten Abschätzung eine [Veränderung/oder nicht] dar.

\section{Ausblick}
%
Die in dieser Arbeit durchgeführte Bestimmung der Normierungskonstante
