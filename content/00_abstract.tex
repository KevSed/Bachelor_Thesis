\thispagestyle{plain}

\section*{Kurzfassung}
Diese Bachelorarbeit befasst sich mit der Bestimmung einer Normierungskonstante für den Zerfall $\signal$ über eine Analyse des Zerfalls $\kontroll$. Dazu wird auf im Jahre 2012 am LHCb-Detektor bei einer integrierten Luminosität von $\SI{2}{\femto\barn^{-1}}$ und einer Schwerpunkstenergie von $\SI{8}{\tera\electronvolt}$ aufgenommene Daten eine Selektion durchgeführt, aus welcher schließlich die Anzahl der Signalkandidaten bestimmt wird. Zusammen mit den während der Analyse bestimmten Effizienzen kann hieraus eine Normierungskonstante $\alpha$ berechnet werden. Der Zerfall $\signal$ ist im Standardmodell verboten, da er die Erhaltung des \textit{lepton-flavor} verletzt. Eine parallel durchgeführte Analyse beschäftigt sich mit der Optimierung der Suche nach Kandidaten für diesen Zerfall. Zusammen mit der Normierungskonstante kann eine obere Abschätzung für die Zerfallsbreite $\mathcal{B}(\signal)$ zu $(9,2\,\pm\,3,4)\cdot10^{-5}$ bestimmt werden.

\section*{Abstract}
\begin{english}
This thesis aims at determining a normalisation factor $\alpha$ for the decay $\signal$, which is strictly forbidden within the standard model of particle physics, as it is violating the conservation of lepton-flavor. To achieve this, the similar decay $\kontroll$ is analysed. Within this analysis the selection of data taken in 2012 at the LHCb detector at a centre of mass energy of $\SI{8}{\tera\electronvolt}$, corresponding to an integrated luminosity of $\SI{2}{\femto\barn^{-1}}$ is optimized, so that a number of expected signal-events can be determined. Taking this number and the efficiencies of the selection into consideration, a normalisation factor is calculated. A simultaneously implemented study aims at the optimization of the search for $\signal$ events. Combining these two studies, an upper limit of $(9,2\,\pm\,3,4)\cdot10^{-5}$ can be set for the branching fraction $\mathcal{B}(\signal)$.
\end{english}
