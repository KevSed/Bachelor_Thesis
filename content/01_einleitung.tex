\chapter{Einleitung}

Die Teilchenphysik beschäftigt sich mit dem Verständnis der Physik auf elementarster Ebene. Dazu gehört die Beschreibung und Vorhersage der Elementarteilchen sowie deren Wechselwirkungen untereinander. Über die letzten Jahrzehnte ist dabei das sogenannte Standardmodell der Teilchenphysik entstanden, welches bis heute die beste Beschreibung in dieser Hinsicht liefert. Dennoch existieren Phänomene, die sich durch das Standardmodell nicht beschreiben lassen: Dunkle Materie, etc sind Beispiele. Die ständige Überprüfung der Vorhersagen also sowie die Suche nach Physik, die über das Standardmodell hinaus geht sind Aufgaben von Teilchenbeschleunigern wie dem LHC (Large Hadron Collider) am CERN. 
