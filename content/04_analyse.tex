\chapter{Analyse des Kontrollkanals}
\label{chap:4}
%
Ziel dieser Analyse ist es die benötigten Größen zur Bestimmung einer Normierungskonstante $\alpha$ für den Signalzerfall $\signal$ zu ermitteln. Diese ist in Gleichung~\eqref{eq:konst} definiert.
%
\begin{equation}
  \alpha=\frac{\mathcal{B}(B\rightarrow\jpsi K)\mathcal{B}(\kontroll)}{N_{\kontroll}}\frac{(\epsilon_{\text{ges}})_{\kontroll}}{(\epsilon_{\text{ges}})_{\signal}}
  \label{eq:konst}
\end{equation}
%

Hierbei ist~$\mathcal{B}(\kontroll)$ das bereits zu $\SI{5,986(33)}{\percent}$ \cite{pdg} vermessene Verzweigungsverhältnis des Kontrollkanals und~$N_{\kontroll}$ die in dieser Arbeit bestimmte Anzahl der gemessenen Signalereignisse für $B\rightarrow\jpsi(\rightarrow\mu\mu) K$.
$\epsilon_{\text{ges}}$ beschreibt die Gesamteffizienzen der Analyse des jeweiligen Kanals. Sie beinhalten die Detektion,
Rekonstruktion und Selektion der Ereignisse. Die Normierungskonstante ermöglicht es, systematische Fehler in der Bestimmung der
Signalkandidaten und Effizienzen des Signalzerfalls $\signal$ möglichst herauszukürzen und so ein deutlich exakteres Ergebnis
zu bestimmen. $\alpha$ kann dazu genutzt werden, zusammen mit der in \cite{ba-maik} ermittelten Anzahl erwarteter
Signalkandidaten eine Abschätzung für das Verzweigungsverhältnis des Zerfalls $\signal$ nach Gleichung~\eqref{eq:absch} zu ermitteln.
%
\begin{equation}
  \mathcal{B}(\signal)<N_{\signal, \SI{95}{\percent}}\cdot\alpha\, .
  \label{eq:absch}
\end{equation}
%
Ideal wäre die Betrachtung eines Datensatzes für $\kontroll$. Hierzu existiert allerdings bisher keine Vorselektion für Daten des LHCb, da sich diese noch in Vorbereitung befindet. Daher wird ein Datensatz für den Zerfall $B\rightarrow\jpsi(\rightarrow\mu\mu) K$ verwendet.
Da die Selektion für die darin enthaltenen Ereignisse aufgrund der Abweichung vom Signalzerfall ebenfalls abweichend ist, müssen bestimmte Korrekturen der erwarteten Ereignisse durchgeführt werden. Diese werden in Abschnitt~\ref{sec:norm} diskutiert.
Um die benötigten Größen aus dieser Analyse zu bestimmen, wird eine schnittbasierte Selektion des Kontrollkanals
durchgeführt, um über die Ermittlung der rekonstruierten Signalereignisse und der dabei auftretenden Effizienzen
die Normierungskonstante bestimmen zu können.
%
\section{Datensätze}
%
Bei dem für die Analyse des Zerfalls $\kontroll$ verwendeten Datensatz handelt es sich um im Jahre 2012 am LHCb-Experiment bei einer Schwerpunkstenergie von $\sqrt{s}=\SI{8}{\tera\electronvolt}$ aufgenommene Daten, die einer integrierten Luminosität von $\SI{2}{\femto\barn^{-1}}$ entsprechen. Dazu ist eine dem Zerfall $\kontroll$ angepasste Selektion (Stripping \texttt{Bu2LLK\_mmLine}) verwendet worden, welche eine Effizienz von $\SI{16,099(21)}{\percent}$ aufweist. Die verwendeten Daten enthalten dabei etwa 5959563 experimentell gemessene Daten, sowie etwa 2210052 in sogenannten Monte-Carlo-Simulationen generierte.\\

Monte-Carlo-Simulationen (MC) stellen eine bewährte Methode dar, um die Physik, die eine Theorie beschreibt und die experimentell gemessenen Daten vergleichbar zu machen. Um die Vorhersage für das Signal eines bestimmten Zerfalls in Form einer experimentellen Messung anzupassen, werden verschiedene Algorithmen zur Simulation der Kollisionen sowie deren Messung verwendet. Ziel ist es, die Kollision, anschließende Zerfälle sowie die Wechselwirkungen mit dem spezifischen Detektor möglichst exakt zu simulieren, sodass ein Signaldatensatz entsteht, welcher unter Anderem zur Unterdrückung des experimentell gemessenen Untergrundes, sowie der Suche nach neuer Physik dient. Bei dem hier verwendeten MC handelt es sich um die Simulation des Zerfalls $B\rightarrow\jpsi(\rightarrow\!\mu\mu) K$. Dabei werden die Proton-Proton-Kollisionen durch das Programm \texttt{PYTHIA 8.1} \cite{pythia} simuliert, welches über komplexe Modelle die entstehenden Teilchen generiert. Zerfallspunkte sowie die Produkte der bei diesen Kollisionen entstehenden $B$-Mesonen werden durch \texttt{EVTGEN} \cite{evtgen} simuliert. Die anschließende Wechselwirkug der Zerfallsprodukte mit dem Detektor konstruiert der Algorithmus \texttt{GEANT4}  \cite{geant4}, sodass die so entstandenen Datensätze durch dieselbe Hardware und Software (Trigger, Rekonstruktion etc.), wie die experimentell am LHCb Gemessenen laufen können. Die Umwandlung der berechneten Ereignisse in die dazu benötigten elektrischen Signale übernimmt die Software \texttt{BOOLE} \cite{boole}. Die Simulation geht dabei von einer Schwerpunktsenergie von $\sqrt{s}=\SI{8}{\tera\electronvolt}$ aus. Um die Simulation den Beschaffenheiten des Detektors anzupassen, ist der Winkelbereich der Zerfallsprodukte auf einen etwas größeren als den realen Akzeptanzbereich des Detektors beschränkt. Hierdurch werden Effekte am Rand des Detektors vernachlässigbar.

\section{Selektion des Kontrollkanals}
%
Die Datensätze für den zu untersuchenden Kontrollkanal $B\rightarrow\jpsi(\rightarrow\mu\mu) K$ werden zunächst schnittbasiert selektiert. Es erfolgt also eine Einschränkung auf den physikalisch für den gesuchten Zerfall relevanten Bereich. Dazu werden die in Tabelle~\ref{tab:strip} aufgeführten Bedingungen an die relevanten Ereignisse in dem Datensatz sowohl an die Daten, als auch an die Simulation gestellt. Diese Selektion orientiert sich dabei an der in der Studie zur Signaloptimierung \cite{ba-maik} angewendeten.

\begin{table}[htb]
  \centering
  \caption{Auf den Datensatz angewendete Selektion.}
  \begin{tabular}{lc}
    \toprule
    Selektionsvariable                & Bedingung      \\
    \midrule
    $m_{\jpsi}$                       & $>\num{2946}<\num{3176}$  \\
    $\mu^{-}:\chi_{\text{IP}}^2$      & $>\num{36}$  \\
    $\mu^{+}:\chi_{\text{IP}}^2$      & $>\num{36}$  \\
    $\mu^{-}:\text{GhostProb}$        & $<\num{0.3}$ \\
    $\mu^{+}:\text{GhostProb}$        & $<\num{0.3}$ \\
    $\jpsi:\text{BKGCAT}$(Nur MC)     & $==\num{0}$  \\
    $\jpsi:\text{DIRA}$               & $>\num{0}$  \\
    \bottomrule
  \end{tabular}
  \label{tab:strip}
\end{table}

Ziel dieser Selektion ist es, bei möglichst hoher Effizienz auf tatsächlich aus dem untersuchten Zerfall stammenden Ereignissen, möglichst viele Ereignisse aus sogenannten Untergrund-Zerfällen auszuschließen. Der bei allen Messungen unweigerlich in den Messungen dokumentierte Untergrund stellt eine Kombination verschiedener Teiluntergünde dar. Diese haben mitunter sehr unterschiedliche Ursachen. Der teilweise rekonstruierte Untergrund beispielsweise wird von Zerfällen induziert, die teilweise die selben Endprodukte wie der Signalzerfall erreichen, von denen allerdings nicht alle detektiert werden. Er überwiegt im unteren Massenfenster. Auch die Fehlerbehaftung der Messungen der einzelnen Detektorkomponenten erzeugt einen systematischen Untergrund, da die Messprozesse zwangsläufig statistischen Schwankungen unterliegen. Einen dritten Beitrag liefert der sogenannte kombinatorische Untergrund, der dadurch entsteht, dass beispielsweise zwei aus unabhängigen Zerfällen stammende Myonen, die zusammen zur Masse des $\jpsi$ kombiniert werden damit fälschlicherweise als Signal identifiziert werden.\\
Es wird zunächst ein Massenfenster von $m_{\jpsi}=\SI{2946}{\mega\electronvolt}$ bis $m_{\jpsi}=\SI{3176}{\mega\electronvolt}$ um die Masse des $\jpsi$-Mesons von $\SI{3096.916(11)}{\mega\electronvolt}$ \cite{pdg} gelegt. Diese invariante Masse wird aus den Viererimpulsen der beiden Tochtermyonen berechnet, sodass durch das Fenster sichergestellt wird, dass diese auch aus einem $\jpsi$-Zerfall stammen. Die anderen Selektionsbedingungen sind aus einer rekursiven Optimierung des Signalzerfalls $\signal$ aus der parallel durchgeführten Studie \cite{ba-maik} übernommen, um eine optimale Vergleichbarkeit zu gewährleisten. Selbstverständlich sind sie für den Zerfall $\kontroll$ an die beiden Myonen angepasst.
Die Bedingungen $\mu^{\pm}:\chi_{\text{IP}}^2$ geben an, wie gut die rekonstruierten Myonen einem Primärvertex zugeordnet werden können. $\chi_{\text{IP}}^2$ ist dabei definiert als der Unterschied zwischen dem $\chi^2$ des Primärvertex' unter Berücksichtigung des beschriebenen
Teilchens und jenem ohne Berücksichtigung desselben \cite{chi2}. Die Variable $\mu^{\pm}:\text{GhostProb}$ gibt an, wie groß die Wahrscheinlichkeit
ist, dass die Identifikation eines Myons fälschlicherweise stattfand, sodass die Selektion diese Wahrscheinlichkeit auf maximal $\SI{30}{\percent}$
beschränkt. $\jpsi:\text{BKGCAT}$ legt fest, dass das in der Simulation rekonstruierte Ereignis auch als Signalereignis erzeugt wurde. Über die Variable $\jpsi:\text{DIRA}$ lässt sich die Übereinstimmung der Richtungen der aus den Vertices rekonstruierten Spuren sowie den Impulsen derselben Teilchen prüfen. Da es sich um den $cosinus$ des Winkels zwischen diesen Richtungen handelt, sollte diese Variable möglichst nahe bei $1$ liegen. Nach Anwenden der so gewählten Selektion auf die Simulation bleiben 1814293 Ereignisse. Die Effizienz dieser Selektion folgt daher nach Gleichung~\eqref{eq:eff_strip1}. Die Fehlerangaben entsprechen dabei den Binomial-Fehlern.
%
\begin{equation}
  \epsilon_\text{Selektion}=\frac{N_\text{MC-events after selection}}{N_\text{MC-events after strip}}=\SI{82.09(3)}{\percent} \, .%\frac{1814293}{2210052}
  \label{eq:eff_strip1}
\end{equation}
%
Des Weiteren werden sogenannte Triggerstufen als Bedingung für die Selektion der relevanten Ereignisse herangezogen. Die Trigger fungieren als Entscheidungskriterien dafür, ob ein Ereignis potentiell interessante Physik bereithält und somit vom Detektor aufgenommen und gespeichert werden soll, oder nicht. Die \texttt{TOS}-Trigger (\textit{triggered on signal}) lösen dabei aus, wenn Teilchen aus Signalzerfällen detektiert werden. Die hier, sowie in der Signalanalyse \cite{ba-maik} verwendeten Trigger gliedern sich in die in Tabelle~\ref{tab:trigger} aufgeführten drei Stufen \texttt{L0}, \texttt{HLT1} und \texttt{HLT2}. Diese sortieren nacheinander Ereignisse heraus, die nicht mindestens eine der aufgeführten Bedingungen innerhalb einer Stufe erfüllen. Daher selektiert jede Stufe feiner als die Vorherige.
%
\begin{table}[htb]
  \centering
  \caption{Auflistung der auf das $\jpsi$-Meson angewandten Trigger.
  Ereignisse, die nicht innerhalb jeder einzelnen Triggerstufe mindestens eine Triggerbedingung erfüllen, werden aussortiert.}
  \begin{tabular}{l}
    \toprule
    \textbf{L0-Trigger}                                 \\
    \quad\texttt{L0MuonDecision\_TOS}              \\
    \quad\texttt{L0HadronDecision\_TOS}            \\
    \quad\texttt{L0ElectronDecision\_TOS}          \\
    \midrule
    \textbf{HLT1-Trigger}                               \\
    \quad\texttt{Hlt1TrackAllL0Decision\_TOS}      \\
    \quad\texttt{Hlt1TrackMuonDecision\_TOS}       \\
    \midrule
    \textbf{HLT2-Trigger}                               \\
    \quad\texttt{Hlt2Topo2BodyBBDTDecision\_TOS}   \\
    \quad\texttt{Hlt2Topo3BodyBBDTDecision\_TOS}   \\
    \quad\texttt{Hlt2Topo4BodyBBDTDecision\_TOS}   \\
    \quad\texttt{Hlt2TopoMu2BodyBBDTDecision\_TOS} \\
    \quad\texttt{Hlt2TopoMu3BodyBBDTDecision\_TOS} \\
    \quad\texttt{Hlt2TopoMu4BodyBBDTDecision\_TOS} \\
    \bottomrule
  \end{tabular}
  \label{tab:trigger}
\end{table}
%
Die Effizienz $\epsilon$ dieser drei Triggerstufen, sowie desr Selektion bezogen auf die Zahl der Ereignisse im Datensatz beträgt für die Simulation die in Gleichung~\eqref{eq:eff_ges1} aufgeführten Wert.
%
\begin{equation}
  \epsilon_\text{Trigger}=\frac{N_\text{MC-events after trigger}}{N_\text{MC-events after selection}}=\SI{74.15(4)}{\percent} \, .% \frac{1345240}{1814293}
  \label{eq:eff_ges1}
\end{equation}
%
Zusammen mit der Generatoreffizienz $\epsilon_\text{gen}=\SI{16.099(21)}{\percent}$ und der Strippingeffizienz $\epsilon_\text{strip}=\SI{8.744(6)}{\percent}$ kann so die Gesamteffizienz wie in Gleichung~\eqref{eq:eff_ges} berechnet werden.
%
\begin{equation}
  \epsilon_{\kontroll, \text{ges}}
  =\epsilon_\text{gen}\cdot\epsilon_\text{strip}\cdot\epsilon_\text{Selektion}\cdot\epsilon_\text{Trigger}=\SI{0.857(1)}{\percent}
  \label{eq:eff_ges}
\end{equation}
%
\section{Massenfit}
%
Um aus dem so eingeschränkten und selektierten MC-Datensatz die Anzahl der Signalkandidaten $N_{\kontroll}$ zu bestimmen, wird ein sogennanter \textit{extended maximum-likelihood-fit} \cite{extended} der rekonstruierten $\jpsi$-Masse durchgeführt. Die oben
beschriebene Selektion beschränkt den Massenbereich des rekonstruierten $\jpsi$-Mesons auf einen Bereich von etwa
$[m_{\jpsi}-\SI{150}{\mega\electronvolt}, m_{\jpsi}+\SI{80}{\mega\electronvolt}]$. Die Masse des $\jpsi$-Mesons beträgt $\SI{3096.916(11)}{\mega\electronvolt}$ \cite{pdg}. Diese Einschränkung konzentriert den Fit auf die $\SI{60.87}{\percent}$ der gesamten Ereignisse, welche in diesem Bereich liegen, da so das gesamte Signal, aber möglichst wenig Untergrund vorliegen. Gleichzeitig ist das Intervall weit genug gewählt, um auch den eingangs dieses Kapitels beschriebenen Untergrund hinreichend modellieren zu können.
%
%\begin{figure}[H]
%  \centering
%      \includegraphics[width=0.9\textwidth]{Plots/jpsi_mass.pdf}
%  \caption{Verteilung der $\jpsi$-Masse nach Anwenden der Selektion sowie der Trigger.}
%  \label{fig:mass}
%\end{figure}
%
Mit Hilfe des frameworks \texttt{RooFit} \cite{roofit} wird diese Verteilung der rekonstruierten $\jpsi$-Masse an ein aus Signalmodell und Untergundmodell bestehendes Gesamtmodell angepasst. Dazu wird für ein aus Signalfunktion und Untergrundfunktion zusammengesetztes Modell der in Gleichung~\ref{eq:massenfit} dargestellten Form eine Ausgleichrechnung durchgeführt.
%
\begin{equation}
  \symup{G}(m_{\jpsi})= \symup{S}(m_{\jpsi}) + \symup{B}(m_{\jpsi})
  \label{eq:massenfit}
\end{equation}
%
$\symup{S}(m_{\jpsi})$ beschreibt hierbei das verwendete Signalmodell, während $\symup{B}(m_{\jpsi})$ den Untergrund modelliert. Die Verteilung der Massen entspricht in etwa einer Gaußverteilung, zeigt aber an den Rändern sogenannte \textit{tails}: nicht der Normalverteilung entsprechende, beispielsweise durch Polynome beschriebende Verteilungen. Diese werden unter Anderem durch die Abstrahlung von niederenergetischen Photonen in Form von \textit{final state radiation} erzeugt \cite{cb}. Um diese Eigenschaft der Verteilung hinreichend zu modellieren, werden sogenannte \textit{Crystal Ball}-Funktionen (CB) verwendet. Diese zur Modellation asymmetrischer Wahrscheinlichkeitsverteilungen verwendeten Funktionen stellen eine Gaußfunktion dar, welche an einer Seite in eine Potenzfunktion übergeht. Die Darstellung ist in Gleichung~\ref{eq:cb} \cite{cb} angegeben, wobei die Parameter $A$ und $B$ stets so bestimmt werden, dass die Funktion analytisch ist.
%
\begin{equation}
  {\displaystyle CB(x;\alpha ,n,{\mu},\sigma,N)=N\cdot {\begin{cases}\exp \left(-{\frac {(x-{\mu})^{2}}{2\sigma ^{2}}}\right),&{\mbox{falls }}{\frac {x-{\mu}}{\sigma }}>-\alpha \\A\cdot \left(B-{\frac {x-{\mu}}{\sigma }}\right)^{-n},&{\mbox{falls }}{\frac {x-{\mu}}{\sigma }}\leqslant -\alpha \end{cases}}}
  \label{eq:cb}
\end{equation}
%
Als Modell für den Untergrund in dieser Messung wird ein exponentieller Zusammenhang der in Gleichung~\ref{eq:exp} dargestellten Form verwendet.
%
\begin{equation}
  B(x; c)= \symup{e}^{c\cdot x} \, .
  \label{eq:exp}
\end{equation}
%
Da durch nicht gaußverteilte Detektoreffekte \textit{tails} auch bei höheren Energien entstehen, kann es sinnvoll sein, durch die Kombination von zwei CB-Funktionen, also einem \textit{Double-Crystal-Ball} (DCB) die Modellation dieser zu ermöglichen \cite{ipatia}.
Im Folgenden wird der Signalteil der Verteilung der rekonstruierten Masse also als DCB-Funktion modelliert. Dies entspricht einem Gesamtmodell der in Gleichung~\ref{eq:dcb} dargestellten Form.
%
\begin{equation}
  \begin{split}
  \symup{G}(m_{\jpsi}; f)={}&\symup{CB}_1(m_{\jpsi};\alpha_1,\mu,\sigma_1,n_1) + f\symup{CB}_2(m_{\jpsi};\alpha_2,\mu,\sigma_2,n_2)\\ &+ \symup{B}(m_{\jpsi};c)
  \end{split}
  \label{eq:dcb}
\end{equation}
%
Der Faktor $f$ beschreibt dabei das Verhältnis der beiden CB-Funktionen und liegt zwischen 0 und 1. Zu dieser Funktion wird der oben beschriebene exponentielle Untergrund ($\symup{B}(m_{\jpsi})$) hinzugefügt und das so entstehende Gesamtmodell an die Daten angepasst. Dazu werden die in Tabelle~\ref{tab:params} aufgeführten Parameter in den angegebenen Intervallen mit Hilfe von \texttt{RooFit} der Verteilung angepasst.
%
%
%\begin{table}[H]
%  \centering
%  \caption{Auflistung der für den Fit des Signalmodells verwendeten Parameter.}
%  \begin{tabular}{ccc}
%    \toprule
%    \textbf{Crystall Ball 1}    & \textbf{Crystall Ball 2} & Intervall für den Fit \\
%    \midrule
%    $\alpha_1$                  & $\alpha_2$               & [-4, 0], [0, 4] \\
%    $\mu$                       & $\mu$                    & [2946, 3176] \\
%    $\sigma_1$                  & $\sigma_2$               & [1, 20], [1, 20] \\
%    $n_1$                       & $n_2$                    & [0, 20], [0, 20] \\
%    \bottomrule
%  \end{tabular}
%  \label{tab:params}
%\end{table}
%
$\alpha$ gibt dabei an, ab wann die Gaußverteilung durch den Potenzzusammenhang ersetzt wird. Dies beeinflusst also, wann die Potenzfunktion
einsetzt. Ein positives $\alpha$ bedeutet dabei, dass der \textit{tail} zu kleineren Werten einsetzt, während ein $\alpha<0$
diesen zu größeren Werten modelliert. $\mu$ beschreibt den Wert für den \textit{peak} der Gaußglocke. Anschaulich beschreibt dies die aus
den Daten rekonstruierte Masse des $\jpsi$-Mesons. Da beide \textit{Crystal Ball}-Funktionen an den gleichen Massenpeak angepasst werden,
wird über diesen Parameter für beide Funktionen gemeinsam iteriert. Das $\sigma$ gibt wie bei Gaußschen Verteilungen üblich die Breite der Verteilung an. Über den Parameter $n$ kann das Modell für den \textit{tail} angepasst werden. Der Wert für $n$ gibt den Grad der Potenz
an.
Die Ausgleichsrechnung ergibt die in Tabelle~\ref{tab:fit1} dargestellten Werte für die freien Parameter. Die so definierte Gesamtfunktion, sowie die einzelnen Komponenten sind in Abbildung~\ref{fig:fit1} dargestellt. Die Wahl mehrerer CB Funktionen ermöglicht es unter anderem, die Gaußverteilung, also den Kern des Fits zu verzerren. Auf diese Weise, ist das Gesamtmodell in der Lage auch nicht gaußverteilte Unsicherheiten in diesem Bereich zu modellieren. Allerdings ist die Ausgleichsrechnung mit \texttt{RooFit} für Modelle dieser Art numerisch sehr instabil, sodass die Wahl der Intervalle und Startwerte für die freien Parameter großen Einfluss auf das Ergebnis hat. Daher ist eine systematische und zuvor genau zu untersuchende Bestimmung dieser von Nöten, welche über den in dieser Arbeit gesteckten Rahmen hinaus ginge. [Was zu Ergebnissen, Fehlern etc...]
%
\begin{table}[H]
  \centering
  \caption{Auflistung der Fit-Ergebnisse des Signalmodells (DCB), sowie des exponentiellen Hintergrunds.}
  \begin{tabular}{lS@{$\,\pm$}S}
    \toprule
    \textbf{Signalmodell}         &  \multicolumn{2}{c}{Fit-Ergebnis} \\
    \midrule
    \quad$\alpha_1$               & 1,41    & 0,04 \\
    \quad$\mu$                    & 3097,79 & 0,02 \\
    \quad$\sigma_1$               & 10,08   & 0,09 \\
    \quad$n_1$                    & 1,49    & 0,07 \\
    \quad$\alpha_2$               & -1,54   & 0,04 \\
    \quad$\mu$                    & 3097,79 & 0,02 \\
    \quad$\sigma_2$               & 13,4    & 0,2 \\
    \quad$n_2$                    & 89      & 50 \\
    \midrule
    \textbf{Untergrundmodell}     &  \multicolumn{2}{c}{Fit-Ergebnis} \\
    \midrule
    \quad$c$                      & -0,008   & 0,002 \\
    \midrule
    \quad$f$                      & 0,52     & 0,01 \\
    \quad$n_{BKG}$                & 4490     & 1394 \\
    \quad$n_{Sig}$                & 1342070  & 1175 \\
    \bottomrule
  \end{tabular}
  \label{tab:fit1}
\end{table}
%
\begin{figure}[H]
  \centering
      \includegraphics[width=1.2\textwidth]{Plots/DCBexp.pdf}
  \caption{Hier kommt eine Verteilung der $\jpsi$-Masse sowie das Ergebnis des DCB Fits in dieser Art (schöner vor endgültiger Abgabe) hin (grün: Untergrund, rot: Gesamtmodell, blau: CB1, orange: CB2).}
  \label{fig:fit1}
\end{figure}
%
%%%%%%%%%%%%%%%%%%%%%%%%%%%%%%%%%%%%%%%%%%%%%%%%%%  IPATIA  %%%%%%%%%%%%%%%%%%%%%%%%%%%%%%%%%%%%%%%%%%%%%%%%%%%%%%%%%%%%%%
Um der oben beschriebenen numerischen Instabilität vorzubeugen, kann beispielsweise die sogenannte \textbf{IPATIA}-Funktion als Signalmodell verwendet werden. Diese ist in Referenz~\cite{ipatia} genau beschrieben. Dazu wird also ein Gesamtmodell der in Gleichung~\ref{eq:ipatia} dargestellten Form verwendet.
%
\begin{equation}
  \begin{split}
  &\symup{I}(m,\mu,\sigma,\lambda,\zeta,\beta,a,n)\propto\\
    &\begin{cases}
      \begin{split}
      ((m-\mu)^2&+A_\lambda²(\zeta)\sigma²)^{\frac{1}{2}\lambda-\frac{1}{4}}\symup{e}^{\beta(m-\mu)}\\
      &\cdot K_{\lambda-\frac{1}{2}}\left(\zeta\sqrt{1+(\frac{m-\ mu}{A_\lambda(\zeta)\sigma})^2}\right),
      \end{split} & \text{wenn} \frac{m-\mu}{\sigma}>-a \\
      \frac{G(\mu-a\sigma,\mu,\sigma,\lambda,\zeta,\beta)}{\left(1-m\/(n\frac{G(\mu-a\sigma,\mu,\sigma,\lambda,\zeta,\beta)}{G'(\mu-a\sigma,\mu,\sigma,\lambda,\zeta,\beta)}-a\sigma)\right)^n} ,& \text{sonst}
    \end{cases}
  \end{split}
  \label{eq:ipatia}
\end{equation}
%
Die Funktion $G(\mu-a\sigma,\mu,\sigma,\lambda,\zeta,\beta)$ ist dabei der für $\frac{m-\mu}{\sigma}>-a$ definierte Zusammenhang. Bei $K$
handelt es sich um Bessel-Funktionen dritter Gattung. Für den hier betrachteten Fall $\zeta=0$ ist auch die Abkürzung $A_\lambda=0$. Das Signalmodell wird über insgesamt neun verschiedene Parameter an die Massenverteilung angepasst. Diese sind in Tabelle~\ref{tab:params} aufgeführt.
%
\begin{table}[H]
  \centering
  \caption{Auflistung der für den Fit des Signalmodells IPATIA verwendeten Parameter.}
  \begin{tabular}{ccc}
    \toprule
    Parameter       & Intervall für den Fit & Startwert \\
    \midrule
    $a$             & [2, 3]                & 2,3 \\
    $a_2$           & [2, 4]                & 2,7 \\
    $n$             & [0,3]                 & 1,3 \\
    $n_2$           & [4, 7]                & 5 \\
    $\beta$         & [-0.1, 0]             & -0,013 \\
    $\zeta$         & (fix)                 & 0 \\
    $\lambda$       & (fix)                 & -2.5 \\
    $\sigma$        & [8, 20]               & 13,61 \\
    \bottomrule
  \end{tabular}
  \label{tab:params}
\end{table}
%
Die Ausgleichsrechnung ergibt die in Tabelle~\ref{tab:fit2} dargestellten Parameter. Die so definierte Gesamtfunktion, sowie die einzelnen Komponenten sind in Abbildung~\ref{fig:fit2} dargestellt. [Fit ist bisher noch nicht konvergiert. Ergebnisse, Fehler etc...]
%
\begin{table}[H]
  \centering
  \caption{Auflistung der Fit-Ergebnisse des Signalmodells (IPATIA), sowie des exponentiellen Hintergrunds.}
  \begin{tabular}{lS@{$\,\pm$}S}
    \toprule
    \textbf{Signalmodell}         &  \multicolumn{2}{c}{Fit-Ergebnis} \\
    \midrule
    \quad$a$                      & 2,4      & 0,02 \\
    \quad$a_2$                    & 4,0      & 0,1  \\
    \quad$n$                      & 1,27     & 0,02 \\
    \quad$n_2$                    & 7,00     & 0,1 \\
    \quad$\beta$                  & -0.0136  & 0,0002 \\
    \quad$\zeta$                  & 0        & 0 \\
    \quad$\lambda$                & -2.5     & 0 \\
    \quad$\sigma$                 & 0,15     & 0,01 \\
    \midrule
    \textbf{Untergrundmodell}     &  \multicolumn{2}{c}{Fit-Ergebnis} \\
    \midrule
    \quad$c$                      & -0,0093   & 0,0005 \\
    \midrule
    \quad$n_\text{BKG}$           & 8000     & 27 \\
    \quad$n_\text{Sig}$           & 1337610  & 1066 \\
    \bottomrule
  \end{tabular}
  \label{tab:fit2}
\end{table}
%
\begin{figure}[H]
  \centering
      \includegraphics[width=\textwidth]{Plots/IPATIAexp.pdf}
  \caption{Hier kommt eine Verteilung der $\jpsi$-Masse sowie das Ergebnis des IPATIA Fits in dieser Art (schöner und mit Legende vor endgültiger Abgabe) hin (grün: Untergrund, rot: Gesamtmodell, blau: IPATIA).}
  \label{fig:fit2}
\end{figure}
%
\section{Bestimmung der Signalkandidaten}
%
Aus den oben durchgeführten Ausgleichsrechnungen folgt, dass sich die Anzahl der Signalkandidaten für $\kontroll$ am exaktesten über das Signalmodell \textsc{IPATIA} bestimmen lässt. Die Anzahl der Signalkandidaten folgt hierbei aus der Ausgleichsrechnung zu: $N_{\kontroll}=1337610\,\pm\,1066$ (Auf Fehler runden?) [wird ergänzt].
Kann auch über $N_{erw}=\mathcal{L}\cdot 2\sigma_{b\bar{b}}\cdot\mathcal{B}(\kontroll)\cdot\epsilon_{\kontroll, ges}$ berechnet werden, aber abhängig von der Genauigkeit der Lumi und des $\sigma_{b\bar{b}}$...
%
\section{Bestimmung der Normierungskonstante}
\label{sec:norm}
%
Aus der parallel durchgeführten Analyse \cite{ba-maik} wird die Gesamteffizienz für die Bestimmung der Signalkandidaten zu
%
\begin{equation}
  \epsilon_{\signal, \text{tot}}=1,2\,\pm\,0,4\cdot 10^{-6}
\end{equation}
%
referenziert. Das Verzweigungsverhältnis für den Zerfall $\kontroll$ ist zu $\SI{5,961(33)}{\percent}$ bestimmt \cite{pdg}. Der in
dieser Analyse verwendete Datensatz unterscheidet sich von dem in der Analyse \cite{ba-maik} verwendeten in der Vorselektion. Die
Kandidaten für den Zerfall $\kontroll$ werden so selektiert, dass die Impulse der beiden Myonen die Masse des $\jpsi$-Mesons
rekonstruieren und dass das $\jpsi$-Meson zusammen mit dem $K^+$ aus dem Zerfall eines $B$-Mesons stammt. Für die Selektion der Kandidaten für $\signal$ werden zwar auch Elektron-Myon-Paare gesucht, deren Impulse die Masse des $\jpsi$-Mesons rekonstruieren, allerdings wird
hier nicht die Rekonstruktion zu einem $B$-Meson vorgenommen. Sie wird dadurch ersetzt, dass der Zerfallsvertex der Myonen nicht direkt
am Kollisionspunkt liegen darf. Um diesen Unterschied in der Selektion auszugleichen, wird als Korrektur für die Anzahl der ermittelten Signalkandidaten der in Gleichung~\eqref{eq:korrektur} definierte, um den Faktor $R$ korrigierte Wert verwendet.
%
\begin{equation}
  N_{\kontroll,\text{korr}}=N_{\kontroll}\cdot\underbrace{\frac{\sigma(pp\rightarrow\jpsi)}{\sigma(pp\rightarrow B^+)}}_{R}
  \label{eq:korrektur}
\end{equation}
%
Um die Zahl der ermittelten Signalkandidaten vergleichbar zu machen, muss die Tatsache, dass die Zahl der produzierten $B$-Mesonen sich von jener der $\jpsi$-Mesonen sich unterscheidet Dabei kommen die sigmas daher.....\\ Der Korrekturfaktor berechnet sich gemäß Gleichung~\eqref{eq:korrfaktor}.
%
\begin{equation}
  R=\frac{\sigma(pp\rightarrow\jpsi)}{\sigma(pp\rightarrow B^+)}=\frac{\SI{12.08(81)}{\micro\barn}}{\SI{44.46(324)}{\micro\barn}}=0,27\,\pm\,0,03
  \label{eq:korrfaktor}
\end{equation}
%
Über den eingangs beschriebenen Zusammenhang~\eqref{eq:konst} für die Normierungskonstante kann diese nun mit Hilfe der bestimmten Werte berechnet werden:
%
\begin{align*}
  \alpha=&\frac{\mathcal{B}(\kontroll)}{N_{\kontroll,\text{korr}}}\frac{(\epsilon_{\text{ges}})_{\kontroll}}{(\epsilon_{\text{ges}})_{\signal}} \\
  =&\frac{\SI{5,961(33)}{\percent}}{1335240\,\pm\,1155}\frac{\SI{0.857(1)}{\percent}}{\SI{1.2(4)e-4}{\percent}}\cdot\frac{1}{0.27\,\pm\,0.03}=0.0012\,\pm0.0004 \, .
\end{align*}
%
\section{Obere Abschätzung für das Verzweigungsverhältnis}
%
Die Normierungskonstante $\alpha$ ermöglicht nun zusammen mit der in der Analyse des Zerfalls $\signal$ bestimmten Anzahl an Signalkandidaten das Bestimmen einer oberen Abschätzung für das Verzweigungsverhältnis $\mathcal{B}(\signal)$. Dieses berechnet sich gemäß des in Gleichung~\eqref{eq:absch} angegebenen Zusammenhanges zu:
%
\begin{equation}
  \begin{split}
    \mathcal{B}(\signal)<N_{\signal, \SI{95}{\percent}}\cdot\alpha=& (7\,\pm\,1)\cdot(0.0012\,\pm0.0004)\\=& 0.0084\,\pm\, \, .
  \end{split}
\end{equation}
%
[Vorerst grobe Abschätzung aus dem Fitergebnis]

\nocite{biblatex, make, toolbox, gitbash, siunitx}
